\documentclass[preprint]{amsart}

% The following \documentclass options may be useful:
%
% 10pt          To set in 10-point type instead of 9-point.
% 11pt          To set in 11-point type instead of 9-point.
% authoryear    To obtain author/year citation style instead of numeric.

\usepackage{ifthen}
\usepackage{enumerate}

\usepackage{xspace}
\usepackage{latexsym,amsmath,amssymb}
\usepackage{amsthm}
%\usepackage{epsfig}
%\usepackage{math-cmds,math-envs}
\usepackage{verbatim}
%\usepackage{proof}
\usepackage{code}
\usepackage{color}
\usepackage{graphics,graphicx}
\usepackage{stmaryrd}
\usepackage{amsbsy}
\usepackage{upgreek}
\usepackage{mathpartir}
\usepackage{url}
\usepackage{mathrsfs}

%\usepackage{unlist}

\usepackage{natbib}
% \bibpunct{[}{]}{,}{n}{}{,}
% \let\cite=\citep

\input{inputs/defs}

% \usepackage{bussproofs}
% \usepackage{unixode}
% \usepackage{xspace}

\newcommand\sound{\operatorname{sound}}
\newcommand\irred{\operatorname{irred}}
\newcommand\ttif{\ensuremath{\mathbf{if}\;}}
\newcommand\keyword[1]{\ensuremath{\mathbf{#1}\;}}
\newcommand\CV{\keyword{CV}}
% \newcommand\FTV{\keyword{FTV}}
\newcommand\STLC{\keyword{STLC}}
\newcommand\fold{\keyword{fold}}
\newcommand\unfold{\keyword{unfold}}
\newcommand\unit{\keyword{unit}}
\newcommand\E{\keyword{E}}
\newcommand\diverge{\keyword{diverge}}
\newcommand\fix{\keyword{fix}}
% \newcommand\FTV{\operatorname{FTV}}
% \newcommand\dom{\operatorname{dom}}
\newcommand\pack{\operatorname{pack}}
\newcommand\unpack{\operatorname{unpack}}
\newcommand\as{\operatorname{as}}
\newcommand\val{\operatorname{val}}


\renewcommand{\slang}{\sfont{\lambda^{1, b, \rightarrow}}}
\newcommand{\stau}{\sfont{\uptau}}


\newcommand{\ssigma}{\sfont{\upsigma}}
\newcommand{\salpha}{\sfont{\upalpha}}
\newcommand{\sdelta}{\sfont{\updelta}}
\newcommand{\smu}{\sfont{\upmu}}
\renewcommand{\stypr}{\sfont{\uptau'}}
\renewcommand{\sb}{\sfont{b}}
\newcommand{\ssf}{\sfont{f}}
\renewcommand{\seif}[3]{\sfont{if~{#1}~{#2}~{#3}}}
\renewcommand{\st}{\sfont{t}}
\newcommand{\styunit}{\sfont{1}}
\newcommand{\styphi}{\sfont{\upphi}}
\newcommand{\sunit}{\sfont{()}}
\newcommand{\sezero}{\sfont{e_0}}

\newcommand{\sGamma}{\sfont{\Gamma}}
\newcommand{\sempctx}{\sfont{[\empctx]}}
\newcommand{\strue}{\sfont{t}}
\newcommand{\sfalse}{\sfont{f}}
\renewcommand{\fold}{\tfont{fold}}
\renewcommand{\unfold}{\tfont{unfold}}
\newcommand{\sctxeq}{\sfont{\approxeq^{ctx}}}
\newcommand{\seq}{\sfont{=^{eq}}}
\newcommand{\sejudg}[3]{\sfont{{#1} \vdash_e {#2} : {#3}}}

\renewcommand{\tlang}{\tfont{\lambda^{1, b, \rightarrow, \mu, \circ, \bullet}}}
\newcommand{\trho}{\tfont{\uprho}}
\newcommand{\ttau}{\tfont{\uptau}}
\renewcommand{\tdelta}{\tfont{\updelta}}
\newcommand{\tsigma}{\tfont{\sigma}}
\newcommand{\ttyunit}{\tfont{1}}
\newcommand{\tb}{\tfont{b}}
\renewcommand{\talpha}{\tfont{\upalpha}}
\newcommand{\tmu}{\tfont{\upmu}}
\newcommand{\trec}[1]{\tfont{\upmu \upalpha.{#1}}}
\newcommand{\tcomp}[2]{\tfont{E^{#1}{#2}}}
\newcommand{\tcir}{\tfont{\circ}}
\newcommand{\tbul}{\tfont{\bullet}}
\newcommand{\tezero}{\tfont{e_0}}
\newcommand{\ttrue}{\tfont{t}}
\newcommand{\tfalse}{\tfont{f}}
\newcommand{\tunit}{\tfont{()}}
\renewcommand{\teif}[3]{\tfont{if~{#1}~{#2}~{#3}}}
\renewcommand{\tefun}[3]{\tfont{\lambda{#1}:{#2}.~{#3}}}
\newcommand{\tGamma}{\tfont{\Gamma}}

\newcommand{\tDelta}{\tfont{\Delta}}
\newcommand{\tempctx}{\tfont{[\empctx]}}
\newcommand{\tempctxv}{\tfont{[\empctx]_v}}
\newcommand{\tejudg}[3]{\tfont{{#1} \vdash_e {#2} : {#3}}}

%%%%%%%%%%%%%%%%%%%%%%%%%%%%%%%%%%%%%%%%%%%%%%%%%%%
%\newcommand{\tepr}{\tfont{\te'}}
\newcommand{\tgamma}{\tfont{\gamma}}
\newcommand{\tctx}{\tfont{\cong^{\mathrm{ctx}}}}
\newcommand{\tciu}{\tfont{\cong^{\mathrm{ciu}}}}
\newcommand{\tciujudg}[4]{\tfont{{#1} \vdash {#2}~\tciu~{#3}~:~{#4}}}
\newcommand{\tE}{\tfont{E}}
\newcommand{\tEjudg}[2]{\tfont{\cdot \vdash \tE~:~(\cdot \rhd {#1}) \leadsto {#2}}}
\newcommand{\tsigmapr}{\tfont{\tsigma'}}
%%%%%%%%%%%%%%%%%%%%%%%%%%%%%%%%%%%%%%%%%%%%%%%%%%
\newcommand{\tprove}[3]{\tfont{{#1}\vdash_e{#2}:{#3}}}
\newcommand{\tprovv}[3]{\tfont{{#1}\vdash_v{#2}:{#3}}}
\newcommand{\trel}[5]{\tfont{{#1}\vdash_v{#2} \simeq^{#3} {#4}:{#5}}}
\newcommand{\dprov}[2]{\tfont{{#1}\vdash{#2}}}
\newcommand{\tsub}[2]{\tfont{{#1}<:{#2}}}
\newcommand{\C}{\tfont{C}}
\newcommand{\Cv}{\tfont{C_v}}

\newcommand{\tctxeq}{\tfont{\approxeq^{ctx}}}
\newcommand{\tlr}{\tfont{\simeq^{lr}}}



\usepackage[yyyymmdd,hhmmss]{datetime}
\usepackage{background}
\backgroundsetup{
  position=current page.east,
  angle=-90,
  nodeanchor=east,
  vshift=-1cm,
  hshift=8cm,
  opacity=1,
  scale=1,
  contents={\textcolor{gray!80}{WORK IN PROGRESS.  DO NOT DISTRIBUTE. (compiled on \today\ at \currenttime)}}
}



\begin{document}
\title{}
\author{}
% \url{shin.hy@husky.neu.edu}\\
% {\small Based on the draft by Daniel Patterson url :
% \url{https://github.ccs.neu.edu/dbp/fabsterm}\\
% and the technical report by Amal Ahmed url:
% \url{http://www.ccs.neu.edu/home/amal/papers/lr-recquant-techrpt.pdf}}


\date{\today}

\maketitle

\begin{definition}[ciu Equivalence]\
  Let $\tejudg{\tGamma}{\te}{\tsigma}$ and $\tejudg{\tGamma}{\tepr}{\tsigma}$.\\
  $\tciujudg{\tGamma}{\te}{\tepr}{\tsigma}$ if
%  \begin{align*}
    $\forall \tgamma, \tE, \tsigmapr$ such that $\tfont{\cdot \vdash \tgamma~:~\tGamma}$ and  $\tEjudg{\tsigma}{\tsigma'}$.\\
    $\tE[\tgamma(\te)] \Downarrow \Rightarrow \tE[\tgamma(\tepr)] \Downarrow$.
%  \end{align*}
\end{definition}

\begin{theorem}[$\tlr$ Completeness with Respect to $\tctx$]\
We show $\tctx \Rightarrow \tciu \Rightarrow \tlr$.
\end{theorem}

\begin{lemma}[$\tctx$ Congruence]\
  If $\tctxjudg{\tGamma}{\te}{\tepr}{\tsigma}$ and $..$
\end{lemma}


\end{document}

\subsection{Target Language $\tlang$}
The target language is more interesting. Its types are split into value, $\ttau$, and computation types, $\tsigma$. The computation types are equipped with syntactic modality indicating purity, $\tcir$, and impurity, $\tbul$. A pure compuation is one that is guaranteed to terminate while an impure computation may or may not terminate. Because of this inclusive behavior, a pure type is a subtype of an impure type and $\tlang$ has the following typing rules:
\begin{gather*}
  \inferrule
  {\tprovv{\tGamma}{\tv}{\ttau}}
  {\tprove{\tGamma}{\tv}{\tcomp{\tcir}{\ttau}}}
  \qquad
  \inferrule
  {\tprove{\tGamma}{\te}{\tcomp{\tcir}{\ttau}}}
  {\tprove{\tGamma}{\te}{\tcomp{\tbul}{\ttau}}}
\end{gather*}
Note that the inversion lemma works on the first rule, but not the second.
\subsubsection{Contexts and Contextual Equivalence $\tctxeq$}\label{context-t}
\subsubsection{Logical relation $\tlr$}
\subsubsection{$\tctxeq$ sound and complete w.r.t $\tlr$}
\subsection{Translation}







\subsection{A fully abstract compiler}



%%%%%%%%%%%%%%%%%%%%%%%%%%%%%%%%%%%%%%%%%%%%%%%%%%%%%%%%%%%%%%%%%%%%%%%%%%
%                             END DOCUMENT                               %
%%%%%%%%%%%%%%%%%%%%%%%%%%%%%%%%%%%%%%%%%%%%%%%%%%%%%%%%%%%%%%%%%%%%%%%%%%
\end{document}

We show a fully abstract compiler from \(\lambda^{1,b,\rightarrow}\) to
\(\lambda^{1,b,\rightarrow,\circ,\bullet}\).

\textbf{Theorem{[}\(\exists\) fully abstract compiler from
\(\lambda^{1,b,\rightarrow}\) to
\(\lambda^{1,b,\rightarrow, \mu, \circ, \bullet}\){]}}

\emph{Proof outline}\\
\({[} (\lambda\^{}\{1,b,\rightarrow\}, \cong\^{}\{ctx\}\_s)
\Leftrightarrow (\lambda\^{}\{1,b,\rightarrow\}, =) {]} {[}
\qquad \qquad \qquad \qquad \qquad \downarrow \qquad \searrow
{]} {[}
\qquad \qquad \qquad \qquad \qquad \qquad \downarrow \qquad \qquad \approx\textsuperscript{\{lr\}
{]} {[} \qquad \qquad \qquad \qquad \qquad \downarrow \qquad \swarrow
{]} {[}
\qquad \qquad \qquad \qquad (\lambda}\{1,b,\rightarrow,\mu,\circ,\bullet\},
\cong\^{}\{ctx\}\_t) {]}\)

In \(\lambda^{1,b,\rightarrow}\), contextual equivalence is equivalent
to \(\beta \eta\) equivalence.\emph{{[}Scherer thesis, Statman 83{]}}
Our proof uses this result to show a contextual equivalence preserving
compiler by showing a \(\beta \eta\) equivalence preserving compiler.

Thus, we will shift our focus from the source language
\(\lambda^{1,b,\rightarrow}\) with contextual equivalence to
\(\lambda^{1,b,\rightarrow}\) with provable equality.

In the latter system, we write the equality between two terms as

{[} \Gamma \vdash e = e' :\sigma {]}

where we assume that \(e\) and \(e'\) has the type \(\sigma\) in context
\(\Gamma\), and we can prove this using a set of axioms and inference
rules. The set of axioms and inference rules also makes provable
equality an equivalence relation and a congruence with respect to the
term-formation operations.

\begin{itemize}
\item
  \(\Gamma \vdash e = e : \sigma \qquad \qquad \qquad \qquad \qquad \qquad\)
  (reflexivity)
\item
  \(\frac{\Gamma \vdash e = e' : \sigma}{\Gamma \vdash e' = e : \sigma} \qquad \qquad \qquad \qquad \qquad \qquad \quad\)
  (symmetry)
\item
  \(\frac{\Gamma \vdash e = e' : \sigma \quad \Gamma \vdash e' = e'' : \sigma}{\Gamma \vdash e = e'' : \sigma} \qquad \qquad \qquad \qquad\)
  (transitivity)
\item
  \(\frac{\Gamma,x:\tau \vdash e = e' : \sigma}{\Gamma \vdash \lambda x:\tau.e = \lambda x:\tau.e' : \tau \rightarrow \sigma} \qquad \qquad \qquad \qquad\)
  (congruence-abs)
\item
  \(\frac{\Gamma \vdash e_f = e_f' : \tau \rightarrow \sigma \quad \Gamma \vdash e_a = e_a' : \tau}{\Gamma \vdash e_f e_a = e_f' e_a' : \sigma} \qquad \quad \qquad\)
  (congruence-app)
\end{itemize}

There are three more axioms: \(\alpha, \beta\), and \(\eta\).
\( \alpha\) equality describes how to construct two equivalent terms by
renaming bound variables. \(\beta\) and \(\eta\) equality specify the
introduction and elimination rules are inverses of each
other.{[}Mitchell{]}

\(\Gamma \vdash \lambda x:\tau.e = \lambda y:\tau.[y/x]e : \tau \rightarrow \sigma\),
provided
\(y \notin FV(M) \qquad \qquad \qquad \qquad \qquad \qquad \qquad \qquad (\alpha)\)

\(\Gamma \vdash (\lambda x:\tau.e)e' = e[e'/x] :\sigma \qquad \qquad \quad \quad \qquad \qquad \qquad \qquad \qquad \qquad \qquad \qquad \qquad \qquad \qquad (\beta)\)

\(\Gamma \vdash \lambda x:\tau.(e x) = e : \tau \rightarrow \sigma\),
provided
\(x \notin FV(e) \qquad \qquad \qquad \qquad \qquad \qquad \qquad \qquad \qquad \qquad (\eta)\)

\begin{center}\rule{0.5\linewidth}{\linethickness}\end{center}

\section{Logical Relation}\label{logical-relation}

\(\Vrel_k[1] \qquad = \{((), ())\}\)\\
\(\mathcal{V}_k[b] \qquad = \{(t,t), (b,b)\}\)\\
\(\mathcal{V}_k[\mu \alpha.\tau] \qquad = \{(\fold v,\fold v') \mid \cdot \vdash \fold v' : \rho^{syn}(\mu \alpha.\tau) \wedge \forall j < k. (v,v') \in \mathcal{V}_j[\tau[\mu \alpha.\tau/\alpha]]\}\)\\
\(\mathcal{V}_k[\tau \rightarrow E^\circ \tau'] = \{(\lambda x:\tau.e, \lambda x:\tau.e') \mid \cdot \vdash \lambda x:\tau.e' : \rho^{syn}(\tau \rightarrow E^\circ \tau') \wedge  \forall j < k.\forall(v,v') \in \mathcal{V}_j[\tau].\)\\
\(\qquad \qquad \qquad \qquad \qquad \qquad \qquad \qquad \qquad \qquad \qquad \qquad \qquad \qquad \qquad \qquad (e[v/x], e'[v'/x]) \in \mathcal{E}_j[E^\circ \tau']\}\)\\
\(\mathcal{V}_k[\tau \rightarrow E^\bullet \tau'] = \{(\lambda x:\tau.e, \lambda x:\tau.e') \mid \cdot \vdash \lambda x:\tau.e' : \rho^{syn}(\tau \rightarrow E^\bullet \tau') \wedge \forall j < k.\forall(v,v') \in \mathcal{V}_j[\tau].\)\\
\(\qquad \qquad \qquad \qquad \qquad \qquad \qquad \qquad \qquad \qquad \qquad \qquad \qquad \qquad \qquad \qquad (e[v/x], e'[v'/x]) \in \mathcal{E}_j[E^\bullet \tau']\}\)

\(\mathcal{E}_k[E^\circ \tau] \qquad = \{(e, e') \mid \exists j \leq k, e_f, e_f'. e \mapsto^j e_f \wedge e' \mapsto^* e_f' \wedge (e_f, e_f') \in \mathcal{V}_{k-j}[\tau]\}\)\\
\(\mathcal{E}_k[E^\bullet \tau] \qquad = \{(e, e') \mid \forall j < k, e_f. e \mapsto^j e_f \wedge \irred(e_f) \Rightarrow \exists e_f'. e' \mapsto^* e_f' \wedge (e_f, e_f') \in \mathcal{V}_{k-j}[\tau]\}\)

\(\mathcal{G}_k[\cdot] \qquad \qquad = \{(\phi, \phi)\}\)\\
\(\mathcal{G}_k[\Gamma,x:\tau] \quad = \{(\gamma[x \mapsto v], \gamma[x \mapsto v']) \mid (\gamma, \gamma') \in \mathcal{G}_k[\Gamma] \wedge (v,v') \in \mathcal{V}_k[\tau]\}\)

\(\Gamma \vdash e \leq_e e' : E^\delta \tau \quad \overset{\text{def}}{=} \quad \Gamma \vdash e:E^\delta  \wedge \Gamma \vdash e':E^\delta \tau \wedge \forall k \geq 0,\gamma,\gamma'. (\gamma,\gamma') \in \mathcal{G}_k[\Gamma] \Rightarrow (\gamma(e), \gamma'(e')) \in \mathcal{E}_k[E^\delta \tau]\)\\
\(\Gamma \vdash e \simeq^{lr}_e e' : E^\delta \tau \quad \overset{\text{def}}{=} \quad \Gamma \vdash e \leq e':E^\delta \tau \wedge \Gamma \vdash e' \leq e:E^\delta \tau\)

\(\Gamma \vdash v \leq_v v' : \tau \quad \overset{\text{def}}{=} \quad \Gamma \vdash v: \tau \wedge \Gamma \vdash v':\tau \wedge \forall k \geq 0,\gamma,\gamma'. (\gamma,\gamma') \in \mathcal{G}_k[\Gamma] \Rightarrow (\gamma(v), \gamma'(v')) \in \mathcal{V}_k[\tau]\)\\
\(\Gamma \vdash v \simeq^{lr}_v v' : \tau \quad \overset{\text{def}}{=} \quad \Gamma \vdash v \leq_v v':\tau \wedge \Gamma \vdash v' \leq_v v:\tau\)

\begin{center}\rule{0.5\linewidth}{\linethickness}\end{center}

\subsubsection{Lemma {[}Values are
well-typed{]}}\label{lemma-values-are-well-typed}

If \((v,v') \in \mathcal{V}_k[\tau]\), then \(\Gamma \vdash v' : \tau\).

\emph{Proof.} By induction on \( \Delta \vdash \tau\).

\begin{center}\rule{0.5\linewidth}{\linethickness}\end{center}

\subsubsection{Lemma {[}Closure under decreasing
step-index{]}}\label{lemma-closure-under-decreasing-step-index}

If \((v,v') \in \mathcal{V}_k[\tau]\) and \(j \leq k\), then
\((v, v') \in \mathcal{V}_j[\tau]\).

\emph{Proof.} By induction on \( \Delta \vdash \tau\).

We are done with the value cases. Now we prove the computation cases.

\subsubsection{\texorpdfstring{Lemma
{[}Compatibility-\(\ttif\){]}}{Lemma {[}Compatibility-\textbackslash{}if{]}}}\label{lemma-compatibility-if}

\(\frac{\Gamma \vdash e_b\leq_e e_b' : E^{\delta_b} b \qquad \Gamma \vdash e_t \leq_e e_t' : E^{\delta_t} \tau \qquad \Gamma \vdash e_f \leq_e e_f' : E^{\delta_f} \tau}{\Gamma \vdash if e_b e_t e_f \leq_e if e_b' e_t' e_f' : E^{\delta_b \wedge \delta_t \wedge \delta_f} \tau}\)

\emph{Proof.}\\
There are two cases to prove:
\(\delta_b \wedge \delta_t \wedge \delta_f = \circ\) and
\(\delta_b \wedge \delta_t \wedge \delta_f = \bullet\).\\
We prove the pure case first and use the {[}Pure implies impure{]} lemma
to prove the impure case. + \textbf{Case
\(\delta_b \wedge \delta_t \wedge \delta_f = \circ \):}\\
In this case it must be true that
\(\delta_b = \delta_t = \delta_f = \circ\).\\
The proof is in two parts.\\
1. We are required to show
\( \Gamma \vdash \ttif e_b e_t e_f : E^\circ b\) and
\( \Gamma \vdash \ttif e_b' e_t' e_f' : E^\circ b\),\\
which follow respectively from\\
- \(\Gamma \vdash e_b : E^\circ b,  \Gamma \vdash e_t : E^\circ \tau,\)
and \( \Gamma \vdash e_f : E^\circ \tau\) and\\
- \(\Gamma \vdash e_b' : E^\circ b, \Gamma \vdash e_t' : E^\circ \tau,\)
and \( \Gamma \vdash e_f' : E^\circ \tau\), both of which follow from\\
+
\(\Gamma \vdash e_b \leq_e e_b' : E^\circ b, \Gamma \vdash e_t \leq_e e_t' : E^\circ \tau\),
and \( \Gamma \vdash e_f \leq_e e_f' : E^\circ \tau\)\\
2. Consider arbitrary \(k, \gamma\), and \(\gamma'\) such that\\
- \(k \geq 0\)\\
- \((\gamma, \gamma') \in \mathcal{G}_k[\Gamma]\)

    We need to show $(\gamma(\ttif e_b e_t e_f), \gamma'(\ttif e_b' e_t' e_f')) \in \mathcal{E}_k[E^\circ \tau]$.
   Note that $(\gamma(\ttif e_b e_t e_f), \gamma'(\ttif e_b' e_t' e_f')) \equiv (\ttif \gamma(e_b) \gamma(e_t) \gamma(e_f), \ttif \gamma'(e_b') \gamma'(e_t') \gamma'(e_f')))$.
   Hence, it suffices to show $(\ttif \gamma(e_b) \gamma(e_t) \gamma(e_f), \ttif \gamma'(e_b') \gamma'(e_t') \gamma'(e_f'))) \in \mathcal{E}_k[E^\circ \tau]$.
   In order to do so, we need to show there exist $j, e_f$ and $e_f'$ such that
    \item $j \leq k$
    \item $\ttif \gamma(e_b) \gamma(e_t) \gamma(e_f)) \mapsto^j e_f$
    \item $\ttif \gamma'(e_b') \gamma'(e_t') \gamma'(e_f')) \mapsto^* e_f'$
    \item $(e_f, e_f') \in \mathcal{V}_{k-j}[\tau]$

   By the operational semantics, we know $\gamma(e_b)$ must reduce first.
   So instantiate $\Gamma \vdash e_b \leq_e e_b' : E^\circ b$ with $k, \gamma$ and $\gamma'$.
   Note that
    \item $k \geq 0$
    \item $(\gamma, \gamma') \in \mathcal{G}_k[\Gamma]$

   Thus, we get $(\gamma(e_b), \gamma'(e_b)) \in \mathcal{E}_k[E^\circ b]$, which implies $\exists j_b, e_{f_b}$, and $e_{f_b}'$ such that
   \item $j_b \leq k$
   \item $\gamma(e_b) \mapsto^{j_b} e_{f_b}$
   \item $\gamma'(e_b) \mapsto^* e_{f_b}'$
   \item $(e_{f_b}, e_{f_b}') \in \mathcal{V}_{k-j_b}[b]$

   Hence, it must be either $e_{f_b} = e_{f_b}'= t$ or $e_{f_b} = e_{f_b}'= f$.

   - **Case $e_{f_b} = e_{f_b}'= t $:**
     Note $\gamma(\ttif e_b e_t e_f) \equiv \ttif \gamma(e_b) \gamma(e_t) \gamma(e_f)$
     $\qquad \qquad \qquad \qquad \mapsto^{j_0} \ttif e_{f_b} \gamma(e_t) \gamma(e_f)$
     $\qquad \qquad \qquad \qquad \equiv \ttif t \gamma(e_t) \gamma(e_f)$
     $\qquad \qquad \qquad \qquad \mapsto^1 \gamma(e_t)$
     $\qquad \qquad \qquad \qquad \mapsto^{j_t} e_{f_1}$
     where $e_{f_1} = e_f'$ and $j= j_b + 1 + j_t$.
     Instantiate $\Gamma \vdash e_t \leq_e e_t' : E^\circ \tau$ with $k-j_b-1$, $\gamma$, and $\gamma'$.
     Note that
     \item $k-j_b-1 \geq 0$  from $j_b < j \leq k$.
     \item $(\gamma, \gamma') \in \mathcal{G}_{k-j_b-1}[\Gamma]$ by the [Context Downward Closed] Lemma applied to $(\gamma, \gamma') \in \mathcal{G}_k[\Gamma]$

     Hence, $(\gamma(e_t), \gamma'(e_t')) \in \mathcal{E}_{k-j_b-1}[E^\circ \tau]$, which implies $\exists j_t, e_{f_t}$, and $e_{f_t}'$ such that
     \item $j_t \leq k-j_b-1$
     \item $\gamma(e_t) \mapsto^{j_t} e_{f_t}$
     \item $\gamma'(e_t') \mapsto^* e_{f_t}'$
     \item $(e_{f_t}, e_{f_t}') \in \mathcal{V}_{k-j_b-1-j_t}[\tau]$

     Let $e_f = e_{f_t}$, and $e_f' = e_{f_t}'$.
     We must show
     \item $j \leq k$, which follows from  $j_b+1+j_t \leq k$
     \item $(e_f, e_f') \in \mathcal{V}_{k-j}[\tau]$, which follows from
       - $(e_{f_t}, e_{f_t}') \in \mathcal{V}_{k-j_b-1-j_t}[\tau]$

   - **Case $e_{f_b} = e_{f_b}'= f $:**
     Note $\gamma(\ttif e_b e_t e_f) \equiv \ttif \gamma(e_b) \gamma(e_t) \gamma(e_f)$
     $\qquad \qquad \qquad \qquad \mapsto^{j_b} \ttif e_{f_b} \gamma(e_t) \gamma(e_f)$
     $\qquad \qquad \qquad \qquad \equiv \ttif f \gamma(e_t) \gamma(e_f)$
     $\qquad \qquad \qquad \qquad \mapsto^1 \gamma(e_f)$
     $\qquad \qquad \qquad \qquad \mapsto^{j_f} \gamma(e_f)$
     where $e_{f_1} = e_f'$ and $j= j_b + 1 + j_f$.
     Instantiate $\Gamma \vdash e_f \leq_e e_f' : E^\circ \tau$ with $k-j_b-1$, $\gamma$, and $\gamma'$.
     Note that
      \item $k-j_b-1 \geq 0$ from $j_b < j \leq k$
      \item $(\gamma, \gamma') \in \mathcal{G}_{k-j_b-1}[\Gamma]$ by the [Context Downward Closed] Lemma applied to $(\gamma, \gamma') \in \mathcal{G}_k[\Gamma]$

     Hence, $(\gamma(e_f), \gamma'(e_f')) \in \mathcal{E}_{k-j_b-1}[E^\circ \tau]$, which implies $\exists j_f, e_{f_f}$, and $e_{f_f}'$ such that
      \item $j_f \leq k-j_b-1$
      \item $\gamma(e_f) \mapsto^j_f e_{f_f}$
      \item $\gamma'(e_f') \mapsto^* e_{f_f}'$
      \item $(e_{f_f}, e_{f_f}') \in \mathcal{V}_{k-j_b-1-j_f}[\tau]$

     Let $j = j_b+1+j_f$, $e_f = e_{f_f}$, and $e_f' = e_{f_f}'$.
     We must show
      \item $j \leq k$, which follows from  $j_b+1+j_f \leq k$
      \item $(e_f, e_f') \in \mathcal{V}_{k-j}[\tau]$, which follows from
        - $(e_{f_f}, e_{f_f}') \in \mathcal{V}_{k-j_b-1-j_f}[\tau]$

\begin{itemize}
\tightlist
\item
  \textbf{Case
  \(\delta_b \wedge \delta_t \wedge \delta_f = \bullet :\)}\\
  By the {[}Pure implies impure{]} Lemma applied to the pure case.
\end{itemize}

\begin{center}\rule{0.5\linewidth}{\linethickness}\end{center}

\subsubsection{Lemma
{[}Compatibility-App{]}}\label{lemma-compatibility-app}

\(\frac{\Gamma \vdash e_f \leq_e e_f' : E^{\delta_f} (\tau \rightarrow E^{\delta_r}\tau') \qquad \Gamma \vdash e_a  \leq_e e_a' : E^{\delta_a} \tau}{\Gamma \vdash e_fe_a  \leq_e e_f'e_a' : E^{\delta_f \wedge \delta_r \wedge \delta_a} \tau'}\)

\emph{Proof}.\\
There are two cases to prove:
\(\delta_f \wedge \delta_r \wedge \delta_a = \circ\) and
\(\delta_f \wedge \delta_r \wedge \delta_a = \bullet\).\\
We prove the pure case first and use the {[}Pure implies impure{]} lemma
to prove the impure case. + \textbf{Case
\(\delta_f \wedge \delta_r \wedge \delta_a = \circ \):}\\
In this case it must be true that
\(\delta_f = \delta_r = \delta_a = \circ\).\\
The proof is in two parts.\\
1. We are required to show \(\Gamma \vdash e_fe_a : E^\circ \tau'\) and
\(\Gamma \vdash e_f'e_a' : E^\circ \tau'\) which follow respectively
from\\
- \(\Gamma \vdash e_f : E^\circ (\tau \rightarrow E^\circ \tau')\) and
\(\Gamma \vdash e_a : E^\circ \tau\) -
\(\Gamma \vdash e_f' : E^\circ (\tau \rightarrow E^\circ \tau')\) and
\(\Gamma \vdash e_a' : E^\circ \tau\) + both of which follow from
\(\Gamma \vdash e_f \leq_e e_f' : E^\circ (\tau \rightarrow E^\circ \tau')\)
and \(\Gamma \vdash e_a \leq_e e_a' : E^\circ \tau\) 2. Consider
arbitrary \(k, \gamma\), and \(\gamma'\) such that\\
- \(k \geq 0\) - \((\gamma, \gamma') \in \mathcal{G}_k[\Gamma]\)

  We are required to show $(\gamma(e_fe_a), \gamma'(e_f'e_a)) \in \mathcal{E}_k[E^\circ \tau']$.
  Note that $(\gamma(e_fe_a), \gamma'(e_f'e_a)) \equiv (\gamma(e_f)\gamma(e_a), \gamma'(e_f')\gamma'(e_a'))$.
  Hence, it suffices to show $(\gamma(e_f)\gamma(e_a), \gamma'(e_f')\gamma'(e_a')) \in \mathcal{E}_k[E^\circ \tau']$.
  In order to do so, we need to show $\exists j, e_f$, and $e_f'$ such that
   - $j \leq k$
   - $\gamma(e_f)\gamma(e_a) \mapsto^j e_f$
   - $\gamma'(e_f')\gamma'(e_a') \mapsto^* e_f'$
   - $(e_f, e_f') \in \mathcal{V}_{k-j}[\tau']$

  By the operational semantics, we know $\gamma(e_f)$ and $\gamma'(e_f')$ must reduce first.
  So instantiate $\Gamma \vdash e_f \leq_e e_f' : E^\circ(\tau \rightarrow E^\circ \tau')$ with $k, \gamma$, and $\gamma'$.
  Note that
    - $k \geq 0$
    - $(\gamma, \gamma') \in \mathcal{G}_k[\Gamma]$

  Hence, we have $(\gamma(e_f), \gamma'(e_f')) \in \mathcal{E}_k[E^\circ(\tau \rightarrow E^\circ \tau')]$, which implies $\exists j_f,e_{f_f},$ and $e_{f_f}'$ such that
    - $j_f \leq k$
    - $\gamma(e_f) \mapsto^{j_f} e_{f_f}$
    - $\gamma'(e_f') \mapsto^* e_{f_f}'$
    - $(e_{f_f}, e_{f_f}') \in \mathcal{V}_{k-j_f}[\tau \rightarrow E^\circ \tau']$

  For convenience, let $e_{f_f} = \lambda x:\tau.e$ and $e_{f_f}' = \lambda x:\tau.e'$ for some $e$ and $e'$.
  By the operational semantics, we know $\gamma(e_a)$ and $\gamma'(e_a')$ must reduce next.
  So instantiate $\Gamma \vdash e_a \leq_e e_a' : E^\circ \tau$ with $k-j_f, \gamma$, and $\gamma'$.
  Note that
   - $k-j_f \geq 0$ from $j_f \leq k$
   - $(\gamma, \gamma') \in \mathcal{G}_{k-j_f}[\Gamma]$ by the [Context downward closed] Lemma applied to $(\gamma, \gamma') \in \mathcal{G}_k[\Gamma]$

  Hence, $(\gamma(e_a), \gamma'(e_a')) \in \mathcal{E}_{k-j_f}[\tau]$, which implies $\exists j_a, e_{f_a}$, and $e_{f_a}'$ such that
    - $j_a \leq k-j_f$
    - $\gamma(e_a) \mapsto^{j_a} e_{f_a}$
    - $\gamma'(e_a') \mapsto^* e_{f_a}'$
    - $(e_{f_a}, e_{f_a}') \in \mathcal{V}_{k-j_f-j_a}[\tau]$

  For convenience, let $e_{f_a} = v_{f_a}$ and $e_{f_a}' = v_{f_a}'$.
  Note that $\gamma(e_fe_a) \equiv \gamma(e_f)\gamma(e_a)$
  $\qquad \qquad \qquad \quad\mapsto^{j_f} e_{f_f}\gamma(e_a)$
    $\qquad \quad \qquad \qquad \equiv (\lambda x:\tau.e) \gamma(e_a)$
    $\qquad \quad \qquad \qquad \mapsto^{j_a} (\lambda x:\tau.e) e_{f_a}$
    $\qquad \quad \qquad \qquad \equiv (\lambda x:\tau.e) v_{f_a}$
    $\qquad \quad \qquad \qquad \mapsto^1 e[v_{f_a}/x]$ by the operational semantics
    $\qquad \quad \qquad \qquad \mapsto^{j_r} v_r$
  where $j = j_f + j_a + 1 + j_r$ and $e_f = v_r$.
  Similarly, $\gamma'(e_f'e_a') \equiv \gamma'(e_f')\gamma'(e_a')$
    $\qquad \qquad \qquad \quad\mapsto^* e_{f_f}'\gamma'(e_a')$
    $\qquad \qquad \qquad \quad \equiv (\lambda x:\tau.e') \gamma'(e_a')$
    $\qquad \qquad \qquad \quad \mapsto^* (\lambda x:\tau.e') e_{f_a}'$
    $\qquad \qquad \qquad \quad \equiv (\lambda x:\tau.e') v_{f_a}'$
    $\qquad \qquad \qquad \quad \mapsto^1 e'[v_{f_a'}/x]$ by the operational semantics
    $\qquad \qquad \qquad \quad \mapsto^* v_r'$
 Instantiate $(\lambda x:\tau.e, \lambda x:\tau.e') \in \mathcal{V}_{k-j_f}[\tau \rightarrow E^\circ \tau']$ with $k-j_f-j_a-1$ and $(v_{f_a}, v_{f_a}') \in \mathcal{V}_{k-j_f-j_a-1}[\tau]$.
 Then, we have $(e[v_{f_a}/x], e'[v_{f_a}'/x]) \in \mathcal{E}_{k-j_f-j_a-1}[E^\circ \tau']$, which implies $\exists j_r, e_{f_r}$, and $e_{f_r}'$ such that
   - $j_r \leq k-j_f-j_a-1$
   - $e[v_{f_a}/x] \mapsto^j_a e_{f_r}$
   - $e'[v_{f_a}'/x] \mapsto^* e_{f_r}'$
   - $(e_{f_r}, e_{f_r}') \in \mathcal{V}_{k-j_f-j_a-1-j_r}[\tau']$

  Let $j=j_f+j_r+1+j_a$, $e_f = e_{f_r}$, and $e_f' = e_{f_r}'$.
  Note that
    - $j_f+j_r+1+j_a \leq k$
    - $\gamma(e_fe_r) \mapsto^{j_f+j_a+1+j_r} e_{f_a}$
    - $\gamma'(e_f'e_a') \mapsto^* e_{f_a}'$
    - $(e_{f_a}, e_{f_a}') \in \mathcal{V}_{k-j_f-j_a-1-j_r}[\tau']$

\begin{itemize}
\tightlist
\item
  \textbf{Case
  \(\delta_f \wedge \delta_r \wedge \delta_a = \bullet :\)}\\
  By the {[}Pure implies impure{]} Lemma applied to the pure case.
\end{itemize}

\subsubsection{\texorpdfstring{Lemma
{[}Compatibility-\(\fold e\){]}}{Lemma {[}Compatibility-\textbackslash{}fold\textasciitilde{}e{]}}}\label{lemma-compatibility-folde}

\(\frac{\Gamma \vdash e \leq_e e' : E^\delta \tau[\mu \alpha.\tau/\alpha]}{\Gamma \vdash \fold e  \leq_e \fold e' : E^\delta \mu \alpha.\tau}\)

\emph{Proof.}\\
There are two cases to prove: \(\delta = \circ\) and
\(\delta = \bullet\).\\
We prove the pure case first and use the {[}Pure implies impure{]} lemma
to prove the impure case.

\begin{itemize}
\tightlist
\item
  \textbf{Case \(\delta = \circ \):}\\
  The proof is in two parts:\\
\end{itemize}

\begin{enumerate}
\def\labelenumi{\arabic{enumi}.}
\tightlist
\item
  We are required to show
  \( \Gamma \vdash \fold e : E^\circ \mu \alpha.\tau\) and
  \( \Gamma \vdash \fold e' : E^\circ \mu \alpha.\tau\) which follow
  respectively from

  \begin{itemize}
  \tightlist
  \item
    \(\Gamma \vdash e : E^\circ \tau[\mu \alpha.\tau/\alpha]\)
  \item
    \(\Gamma \vdash e' : E^\circ \tau[\mu \alpha.\tau/\alpha]\)\\
    both of which follow from
    \(\cdot \vdash e \leq_v e' : E^\circ \tau[\mu \alpha.\tau/\alpha]\)
  \end{itemize}
\item
  Consider arbitrary \(k, \gamma, \gamma'\) such that

  \begin{itemize}
  \tightlist
  \item
    \(k \geq 0\)
  \item
    \((\gamma, \gamma') \in \mathcal{G}_k[\Gamma]\)
  \end{itemize}

  We are required to show
  \((\gamma(\fold e), \gamma'(\fold e')) \in \mathcal{E}_k[E^\circ \mu \alpha.\tau]\).\\
  Note
  \((\gamma(\fold e), \gamma'(\fold e')) \equiv (\fold \gamma (e), \fold \gamma'(e'))\).\\
  Hence, it suffices to show
  \((\fold \gamma(e), \fold \gamma'(e'))\in \mathcal{E}_k[E^\circ \mu \alpha.\tau]\).\\
  In order to do so, we need to show \(\exists j, e_f\), and \(e_f'\)
  such that

  \begin{itemize}
  \tightlist
  \item
    \(j \leq k\)
  \item
    \(\fold \gamma(e) \mapsto^j e_f\)
  \item
    \(\fold \gamma'(e') \mapsto^* e_f'\)
  \item
    \((e_f, e_f') \in \mathcal{V}_{k-j}[\mu \alpha.\tau]\)
  \end{itemize}

  By the operational semantics, we know \(\gamma(e)\) must reduce
  first.\\
  Instantiate
  \(\Gamma e \leq_e e' : E^\circ \tau[\mu \alpha.\tau/\alpha]\) with
  \(k, \gamma\) and \(\gamma'\).\\
  Note that

  \begin{itemize}
  \tightlist
  \item
    \(k \geq 0\)
  \item
    \((\gamma, \gamma') \in \mathcal{G}_k[\Gamma]\)
  \end{itemize}

  Hence, we have
  \((\gamma(e), \gamma'(e')) \in \mathcal{E}_k[E^\circ \tau[\mu \alpha.\tau/\alpha]]\).\\
  It follows that there exists \(j_0, e_{f_0}\), and \(e_{f_0}'\) such
  that

  \begin{itemize}
  \tightlist
  \item
    \(j_0 \leq k\)
  \item
    \(\gamma(e) \mapsto^{j_0} e_{f_0}\)
  \item
    \(\gamma'(e') \mapsto^* e_{f_0}'\)
  \item
    \((e_{f_0}, e_{f_0}') \in \mathcal{V}_{k-j_0}[\tau[\mu \alpha.\tau/\alpha]]\)
  \end{itemize}

  For convenience, let \(e_{f_0} = v_0\) and \(e_{f_0}' = v_0'\).\\
  Note that \(\gamma(\fold e) \equiv \fold \gamma(e)\)\\
  \(\qquad \qquad \qquad \qquad \mapsto^j_0 \fold e_{f_0}\)\\
  \(\qquad \qquad \qquad \qquad \equiv \fold v_0\)\\
  \(\qquad \qquad \qquad \qquad \mapsto^{j-j_0} e_f\)\\
  Since \(\fold v\) is a value, \(j=j_0\) and \(e_f=\fold v_0\).\\
  Let \(e_f' = \fold v_0'\).\\
  We are required to show
  \((\fold v_0, \fold v_0') \in \mathcal{V}_{k-j}[\mu \alpha.\tau]\).\\
  In order to do so, we need to show

  \begin{itemize}
  \tightlist
  \item
    \(\cdot \vdash \fold v_0': \mu \alpha.\tau\)

    \begin{itemize}
    \tightlist
    \item
      which follows from
      \(\cdot \vdash v_0' : \tau[\mu \alpha.\tau/\alpha]\)
    \item
      which follows from
      \((v_0,v_0') \in \mathcal{V}_{k-j}[\tau[\mu \alpha.\tau/\alpha]]\)
    \end{itemize}
  \item
    Consider arbitrary \(i < k-j\).\\
    We are required to show
    \((v_0,v_0') \in \mathcal{V}_i[\tau[\mu \alpha.\tau/\alpha]]\),
    which follows from the {[}Closure under decreasing step index{]}
    lemma applied to
    \((v_0,v_0') \in \mathcal{V}_{k-j}[\tau[\mu \alpha.\tau/\alpha]]\).
  \end{itemize}
\end{enumerate}

\begin{itemize}
\tightlist
\item
  \textbf{Case \(\delta = \bullet :\)}\\
  By the {[}Pure implies impure{]} lemma applied to the pure case.
\end{itemize}

\subsubsection{\texorpdfstring{Lemma
{[}Compatibility-\(\unfold\){]}}{Lemma {[}Compatibility-\textbackslash{}unfold{]}}}\label{lemma-compatibility-unfold}

\(\frac{\Gamma \vdash e \leq_e e' : E^\delta \mu \alpha.\tau}{\Gamma \vdash \unfold e  \leq_e \unfold e' : E^\bullet (\tau[\mu \alpha.\tau/\alpha])}\)

\emph{Proof.}\\
The proof is in two parts: 1. We are required to show
\( \Gamma \vdash \unfold e : E^\bullet(\tau[\mu \alpha.\tau/\alpha])\)
and
\( \Gamma \vdash \unfold e' : E^\bullet(\tau[\mu \alpha.\tau/\alpha])\)
which follow respectively from -
\(\Gamma \vdash e : E^\delta \mu \alpha.\tau\) -
\(\Gamma \vdash e' : E^\delta \mu \alpha.\tau\)\\
both of which follow from
\(\Gamma \vdash e \leq_e e' : E^\delta \mu \alpha.\tau\) 2. Consider
arbitrary \(k, \gamma, \gamma'\) such that - \(k \geq 0\) -
\((\gamma, \gamma') \in \mathcal{G}_k[\Gamma]\)

We are required to show
\((\gamma(\unfold e), \gamma'(\unfold e')) \in \mathcal{E}_k[E^\bullet(\tau[\mu \alpha.\tau/\alpha])]\).\\
Note that
\((\gamma(\unfold e), \gamma'(\unfold e')) \equiv (\unfold \gamma (e), \unfold \gamma'(e'))\).\\
Hence, it suffices to show
\((\unfold \gamma (e), \unfold \gamma'(e')) \in \mathcal{E}_k[E^\bullet (\tau[\mu \alpha.\tau/\alpha])]\).\\
Pick arbitrary \(j\) and \(e_f\) such that\\
- \(j < k\) - \(\unfold \gamma(e) \mapsto^j e_f\)\\
- \(\irred(e_f)\)

We need to show \(\exists e_f'\) such that\\
- \(\unfold \gamma'(e') \mapsto^* e_f'\) -
\((e_f, e_f') \in \mathcal{V}_{k-j}[\tau[\mu \alpha.\tau/\alpha]]\)

By the operational semantics, we know \(\exists j_0\) and \(e_{f_0}\)
such that\\
- \(\gamma(e) \mapsto^{j_0} e_{f_0}\) - \(\irred(e_{f_0})\)

Instantiate \(\Gamma \vdash e \leq_e e' : E^\delta \mu \alpha.\tau\)
with \(k, \gamma\) and \(\gamma'\).\\
Note that\\
- \(k \geq 0\) - \((\gamma, \gamma') \in \mathcal{G}_k[\Gamma]\)

Hence, we have
\((\gamma(e), \gamma(e')) \in \mathcal{E}_k[E^\delta \mu \alpha.\tau]\).\\
There are two cases to consider : \(\delta = \circ\) and
\(\delta = \bullet\).\\
+ \textbf{Case \(\delta = \circ :\)}\\
In this case, we have
\((\gamma(e), \gamma(e')) \in \mathcal{E}_k[E^\circ \mu \alpha.\tau]\).\\
Instantiate this with \(j_0\) and \(e_{f_0}\). Note that\\
- \(j_0 \leq k\) - \(\gamma(e) \mapsto^{j_0} e_{f_0}\)

\begin{verbatim}
   Also note that there exists $e_{f_0}'$ such that
    - $\gamma'(e') \mapsto^* e_{f_0}'$
    - $(e_{f_0}, e_{f_0}') \in \mathcal{V}_{k-j_0}[\mu \alpha.\tau]$

   For convenience, let $e_{f_0} = \fold v_0$ and $e_{f_0}' = \fold v_0'$.
   Note that $\gamma(\unfold e) \equiv \unfold \gamma(e)$
   $\qquad \qquad \qquad \qquad \quad \mapsto^{j_0} \unfold e_{f_0}$
   $\qquad \qquad \qquad \qquad \quad \equiv \unfold (\fold v_0)$
   $\qquad \qquad \qquad \qquad \quad \mapsto^1 v_0$
   Since $v_0$ is a value, $j = j_0 + 1$ and $e_f = v_0$.
   Let $e_f' = v_0'$.
   We are required to show
    + $\unfold \gamma'(e') \mapsto^* v_0'$
      - which follows from $\unfold \gamma'(e') \mapsto^* \unfold e_{f_0}'$
      $\qquad \qquad \qquad \qquad \qquad \qquad \quad \equiv \unfold (\fold v_0')$
      $\qquad \qquad \qquad \qquad \qquad \qquad \quad \mapsto^1 v_0'$
   + $(v_0, v_0') \in \mathcal{V}_{k-j}[\tau[\mu \alpha.\tau/\alpha]]$
     - which follows from the following:
       From $(e_{f_0}, e_{f_0}') \equiv (\fold v_0, \fold v_0') \in \mathcal{V}_{k-j_0}[\mu \alpha.\tau]$, we have
       + $\cdot \vdash \fold v_0' : \mu \alpha.\tau$
       + $\forall i < k-j_0. (v_0,v_0') \in \mathcal{V}_i[\tau[\mu \alpha.\tau/\alpha]]$

         Since $k-j < k-j_0$, $(v_0, v_0') \in \mathcal{V}_{k-j}[\tau[\mu \alpha.\tau/\alpha]]$.
 + **Case $\delta = \bullet :$**
   In this case, we have $(\gamma(e), \gamma(e')) \in \mathcal{E}_k[E^\bullet \mu \alpha.\tau]$.
   Instantiate this with $j_0$ and $e_{f_0}$.
   Note that
    - $j_0 \leq k$
    - $\gamma(e) \mapsto^{j_0} e_{f_0}$

   Hence, we have $e_{f_0}'$ such that
   - $\gamma'(e') \mapsto^* e_{f_0}'$
   - $(e_{f_0}, e_{f_0}') \in \mathcal{V}_{k-j_0}[\mu \alpha.\tau]$

   For convenience, let $e_{f_0} = \fold v_0$ and $e_{f_0}' = \fold v_0'$.
 Note that $\gamma(\unfold e) \equiv \unfold \gamma(e)$
 $\qquad \qquad \qquad \qquad \quad \mapsto^{j_0} \unfold e_{f_0}$
 $\qquad \qquad \qquad \qquad \quad \equiv \unfold (\fold v_0)$
 $\qquad \qquad \qquad \qquad \quad \mapsto^1 v_0$
 Since $v_0$ is a value, $j = j_0 + 1$ and $e_f = v_0$.
 Let $e_f' = v_0'$.
 We are required to show
   + $\unfold \gamma'(e') \mapsto^* v_0'$
     - which follows from $\unfold \gamma'(e') \mapsto^* \unfold e_{f_0}'$
     $\qquad \qquad \qquad \qquad \qquad \qquad \quad \equiv \unfold (\fold v_0')$
     $\qquad \qquad \qquad \qquad \qquad \qquad \quad \mapsto^1 v_0'$
   + $(v_0, v_0') \in \mathcal{V}_{k-j}[\tau[\mu \alpha.\tau/\alpha]]$
     - which follows from the following:
     From $(e_{f_0}, e_{f_0}') \equiv (\fold v_0, \fold v_0') \in \mathcal{V}_{k-j_0}[\mu \alpha.\tau]$, we have
       + $\cdot \vdash \fold v_0' : \mu \alpha.\tau$
       + $\forall i < k-j_0. (v_0,v_0') \in \mathcal{V}_i[\tau[\mu \alpha.\tau/\alpha]]$

        Since $k-j < k-j_0$, $(v_0, v_0') \in \mathcal{V}_{k-j}[\tau[\mu \alpha.\tau/\alpha]]$.
\end{verbatim}

\subsubsection{Lemma {[}Substitutivity{]}}\label{lemma-substitutivity}

If \( \Gamma \vdash v \leq_v v' : \tau\) and
\( \Gamma,x:\tau \vdash e \leq_e e': E^\delta \tau'\), then
\( \Gamma \vdash e[v/x] \leq_e e'[v'/x] : E^\delta \tau'\).

\emph{Proof.}\\
The proof is in two parts.\\
1. We are required to show \(\Gamma \vdash e[v/x] : E^\delta \tau'\) and
\(\Gamma \vdash e'[v'/x] : E^\delta \tau'\), which follow respectively
from applying the {[}Substitution{]} lemma to + \(\Gamma \vdash v:\tau\)
and \(\Gamma,x:\tau \vdash e:E^\delta \tau'\)\\
+ \(\Gamma \vdash v':\tau\) and
\(\Gamma,x:\tau \vdash e':E^\delta \tau'\)\\
- both of which follow from \(\Gamma \vdash v \leq_v v' : \tau\) and
\(\Gamma,x:\tau \vdash e \leq_e e' : E^\delta \tau'\)

\begin{enumerate}
\def\labelenumi{\arabic{enumi}.}
\setcounter{enumi}{1}
\tightlist
\item
  Consider arbitrary \(k, \gamma\), \(\gamma'\) such that

  \begin{itemize}
  \tightlist
  \item
    \(k \geq 0\)
  \item
    \((\gamma, \gamma') \in \mathcal{G}_k[\Gamma]\)
  \end{itemize}

  We must show
  \(\gamma(e[v/x], \gamma'(e'[v'/x])) \in \mathcal{E}_k[E^\delta \tau']\).\\
  There are two cases to consider:

  \begin{itemize}
  \item
    \textbf{Case \(\delta = \circ :\)}\\
    In this case, we are required to show
    \((\gamma(e[v/x], \gamma'(e'[v'/x])) \in \mathcal{E}_k[E^\circ \tau']\).\\
    In order to do so, we need to show \(\exists j, e_f\) and \(e_f'\)
    such that

    \begin{itemize}
    \tightlist
    \item
      \(j \leq k\)
    \item
      \(\gamma(e[v/x]) \mapsto^j e_f\)
    \item
      \(\gamma'(e'[v/x]) \mapsto^* e_f'\)
    \item
      \((e_f, e_f') \in \mathcal{V}_{k-j}[\tau']\)
    \end{itemize}

    Instantiate \(\Gamma \vdash v \leq_v v' : \tau\) with \(k, \gamma\),
    and \(\gamma'\).\\
    Note that

    \begin{itemize}
    \tightlist
    \item
      \(k \geq 0\)
    \item
      \((\gamma, \gamma') \in \mathcal{G}_k[\Gamma]\)
    \end{itemize}

    Hence, we have
    \((\gamma(v), \gamma'(v')) \in \mathcal{V}_k[\tau]\).\\
    Instantiate \(\Gamma,x:\tau \vdash e \leq_e e' : E^\circ \tau'\)
    with \(k, \gamma[x \mapsto \gamma(v)]\), and
    \(\gamma'[x \mapsto \gamma'(v')]\).\\
    Note that

    \begin{itemize}
    \tightlist
    \item
      \(k \geq 0\)
    \item
      \((\gamma[x \mapsto \gamma(v)], \gamma'[x \mapsto \gamma'(v')]) \in \mathcal{G}_k[\Gamma,x:\tau]\)
      which follow from

      \begin{itemize}
      \tightlist
      \item
        \((\gamma, \gamma') \in \mathcal{G}_k[\Gamma]\)\\
      \item
        \((\gamma(v), \gamma'(v')) \in \mathcal{V}_k[\tau]\)
      \end{itemize}
    \end{itemize}

    Hence, we have
    \((\gamma[x \mapsto \gamma(v)](e), \gamma'[x \mapsto \gamma'(v')](e)) \in \mathcal{E}_k[E^\circ \tau']\)\\
    \(\qquad \qquad \quad \equiv (\gamma(e[\gamma(v)/x]), \gamma'(e'[\gamma'(v')/x])) \in \mathcal{E}_k[E^\circ \tau']\)\\
    \(\qquad \qquad \quad \equiv (\gamma(e[v/x]), \gamma'(e'[v'/x])) \in \mathcal{E}_k[E^\circ \tau']\).
  \item
    \textbf{Case \(\delta = \bullet :\)}\\
    By the {[}Pure implies impure-closed{]} lemma applied to the pure
    case.
  \end{itemize}
\end{enumerate}
\end{document}
